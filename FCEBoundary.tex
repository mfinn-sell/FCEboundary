\documentclass[11pt]{amsart}

%Packages.

\usepackage[parfill]{parskip}
\usepackage{verbatim}
\usepackage{amsmath}
\usepackage{amssymb}
\usepackage{amsfonts}
\usepackage{amsthm}
\usepackage{mathrsfs}
\usepackage{tikz}
\usetikzlibrary{matrix,arrows}
\usepackage[all]{xy}
\usepackage{mathrsfs}
\usepackage{a4wide}
\usepackage{enumerate}
\usepackage{hyperref}
%\usepackage{showkeys}

%Theorem styles.

\theoremstyle{plain}
\newtheorem{conj}{Conjecture}
\newtheorem{theorem}{Theorem}%[section]
\newtheorem{lemma}[theorem]{Lemma}%[section]
\newtheorem{proposition}[theorem]{Proposition}%[section]
\newtheorem{corollary}[theorem]{Corollary}%[section]
\newtheorem{claim}[theorem]{Claim}%[section]
\newtheorem{question}[theorem]{Question}
\newtheorem{conjecture}[theorem]{Conjecture}
\newtheorem*{conjecture1}{Conjecture}
\newtheorem{rem}[theorem]{Observation / Remark}
\newtheorem*{theorem1}{Theorem}
\newtheorem*{corollary1}{Corollary}

\theoremstyle{definition}%[section]
\newtheorem{definition}[theorem]{Definition}%[section]
\newtheorem{example}[theorem]{Example}%[section]
\newtheorem{project}{Project}

\theoremstyle{remark}%[section]
\newtheorem{remark}[theorem]{Remark}%[section]

%\numberwithin{theorem}{section}

\newcommand{\G}{\mathcal{G}}
\newcommand{\E}{\widehat{E}}
\newcommand{\X}{\partial \beta \square H}
\newcommand{\A}{\widehat{A}}



\newcommand{\ucong}{\rotatebox{90}{$\cong$}}

\title{Fibred coarse embeddings, a-T-menability and the coarse analogue of the Nokikov conjecture.} 
\date{April 2013}
\author{Martin Finn-Sell}
\email{Martin Finn-Sell: ms1205@soton.ac.uk}

\begin{document}
\bibliographystyle{alpha}

\begin{abstract}
The connection between the coarse geometry of metric spaces and analytic properties of topological groupoids is well known. One of the main results of of Skandalis, Tu and Yu is that a space admits a coarse embedding into Hilbert space if and only if to the associated topological groupoid has the Haagerup property, from where it follows that the coarse Baum-Connes conjecture holds for such a space. The property of a admitting a fibred coarse embedding into Hilbert space was introduced by Chen, Wang and Yu to provide a property that is sufficient for the maximal analogue to the coarse Baum-Connes conjecture and in this paper we connect this property to the traditional coarse Baum-Connes conjecture by constructing a groupoid, similar to the coarse groupoid, that has the Haagerup property if and only if the space admits a fibred coarse embedding into Hilbert space. Additionally, we use this result to give a characterisation of the Haagerup property for residually finite discrete groups.
\end{abstract}

\maketitle

\section{Introduction}

The application of coarse methods to algebraic topological problems is well known \cite{MR866507,MR1728880}. The primary method utilised for such problems is a \textit{higher index} that allows us to calculate refined large scale information from small scale topological or analytic data. This process can succinctly phrased using the language of K-theory and K-homology and can be encoded completely into the coarse geometric version of the well-known Baum-Connes conjecture, which asks whether or not a certain assembly map:
\begin{equation*}
\mu_{*}:KX_{*}(X)=\lim_{R>0}K_{*}(P_{R}(X)) \longrightarrow K_{*}(C^{*}(X))
\end{equation*}
is an isomorphism for all uniformly discrete spaces $X$ with bounded geometry. One well known approach to this conjecture for a suitable metric spaces $X$ is via the concept of a \textit{coarse embedding into Hilbert space}. A seminal paper by Yu \cite{MR1728880} first showed the importance of a coarse embeddings into Hilbert space by proving that this is a sufficient condition for the coarse Baum-Connes assembly map to be an isomorphism.

In this paper we study the relationship between \textit{fibred coarse embedding into Hilbert space}, first introduced by Chen, Wang and Yu \cite{FCEpaper}, and the coarse analogue of the strong Novikov conjecture. Intuitively, a space admits a fibred coarse embedding into Hilbert space if for each scale it is acceptable to forget bounded portions of the space and embed what remains locally into Hilbert space in a compatable manner; this is made precise in Definition \ref{def:FCE}. This property was used in \cite{FCEpaper} to prove a maximal analogue of the result of Yu \cite{MR1728880} described above, that is a fibred coarse embedding into Hilbert space implies that the maximal coarse Baum-Connes assembly map is an isomorphism for any uniformly discrete metric space with bounded geometry.

Another approach to these assembly problems was considered in \cite{mypub1}, in which a conjecture known as the \textit{boundary coarse Baum-Connes conjecture} was defined for uniformly discrete bounded geometry spaces. This conjecture, defined via groupoids, was designed to capture the structure of a space at infinity. We explain the relationship between this conjecture and the work of \cite{FCEpaper} and give a new method to prove the maximal coarse Baum-Connes conjecture in this instance that has additional consequences. We explore these consequences in Section \ref{sect:apps}.

The strategy of this paper is to convert those assembly problems in which we are interested into the world of groupoids as was pioneered by \cite{MR1905840}. We construct a new groupoid $\mathcal{M}(X)$ that can be associated to any sequence of bounded subsets of $X$. This groupoid then plays an important role in characterising fibred coarse embeddability, that is we prove the analogue of Theorem 5.4 from \cite{MR1905840} for a fibred coarse embeddings: 

\begin{theorem}\label{Thm:MR1}
Let $X$ be a uniformly discrete bounded geometry metric space and let $\mathcal{F}:=\lbrace K_{R} \rbrace$ be a nested family of bounded subsets of $X$. Then there exists a groupoid $\mathcal{M}_{\mathcal{F}}(X)$ associated to the family $\mathcal{F}$ such that the following are equivalent:
\begin{enumerate}
\item $X$ admits a fibred coarse embedding into Hilbert space with respect to the family $\mathcal{F}$;
\item$\mathcal{M}_{\mathcal{F}}(X)$ has the Haagerup property.
\end{enumerate}
\end{theorem}
As a natural corollary of this result we can conclude information about the boundary groupoid:
\begin{corollary}\label{Thm:MT1}
Let $X$ be a uniformly discrete space with bounded geometry that admits a fibred coarse embedding into Hilbert space. Then the associated boundary groupoid $G(X)|_{\partial\beta X}$ has the Haagerup property.
\end{corollary}

This result gives us access to the tools developed in \cite{mypub1} concerning the boundary coarse Baum-Connes conjecture and in in Section \ref{sect:apps} we provide applications of Corollary \ref{Thm:MT1} to the coarse analogue of the strong Novikov conjecture for uniformly discrete spaces with bounded geometry (Theorem \ref{thm:mcor1}), as well as a question of Roe concerning ghost operators within Roe algebras associated to coarsely embeddable spaces (Corollary \ref{thm:IT3}). Lastly we give a characterisation of the Haagerup property for finitely generated residually finite groups via box spaces (Theorem \ref{thm:cor2}).

We recap now the important details surrounding Theorem 5.4 from \cite{MR1905840} and the notion of a fibred coarse embedding \cite{FCEpaper}.

\section*{Acknowledgements}
The author would like to thank his supervisor, Nick Wright, for reminding him of the Riesz representation theorem.

\section{Fibred coarse embeddings and groupoids associated to coarse spaces.}\label{sect:coarse}
Throughout this section $X$ will denote a metric space with a proper metric, unless otherwise stated.

\subsection{Coarse and fibred coarse embeddings.}
We first recall the concept of a coarse embedding and motivate the definition of a fibred coarse embedding.

\begin{definition}\label{def:FCE}
A metric space $X$ is said to admit a coarse embedding into Hilbert space $\mathcal{H}$ if there exist maps $f:X \rightarrow \mathcal{H}$,  and non-decreasing $\rho_{1},\rho_{2}:\mathbb{R}_{+} \rightarrow \mathbb{R}$ such that:
\begin{enumerate}
\item for every $x,y \in X$, $\rho_{1}(d(x,y)) \leq \Vert f(x) - f(y) \Vert \leq \rho_{2}(d(x,y))$;
\item for each $i$, we have $\lim_{r \rightarrow \infty}\rho_{i}(r) = +\infty$.
\end{enumerate}
\end{definition}

The primary application of a coarse embedding into Hilbert space is the main result of \cite{MR1728880, MR1905840}.

\begin{theorem}
Let $X$ be a uniformly discrete metric space with bounded geometry that admits a coarse embedding into Hilbert space. Then the coarse Baum-Connes conjecture holds for $X$, that is the assembly map $\mu_{*}$ is an isomorphism.\qed
\end{theorem}

Many metric spaces admit coarse embeddings; the property is a suitably flexible one as it is implied by many other coarse properties such as finite asymptotic dimension \cite{MR1739727}, or the weaker properties of finite decomposition complexity \cite{MR2947546} and property A \cite{MR1728880}. The primary application is to finitely generated discrete groups via the well known descent principle in coarse geometry:

\begin{corollary}
Let $\Gamma$ be a finitely generated discrete group admitting a coarse embedding into Hilbert space. Then the (strong) Novikov conjecture holds for $\Gamma$.\qed
\end{corollary}

For more information on the strong Novikov conjecture and its connections to coarse geometry see \cite{MR1388300, MR1905840}.

It is well known that there are spaces that do not admit a coarse embedding into Hilbert space such as expander graphs \cite{MR2569682}. Certain types of expander graph are known to be counterexamples to the Baum-Connes conjecture \cite{higsonpreprint,MR1911663,explg1,explg2,MR2568691}, but it is known that they do satisfy the coarse analogue of the Novikov conjecture and the maximal coarse Baum-Connes conjecture. The notion of a fibred coarse embedding was introduced in \cite{FCEpaper} where the authors gave results that partially explained these phenomena.

The following is Definition 2.1 from \cite{FCEpaper}.

\begin{definition}
A metric space $X$ is said to admit a \textit{fibred coarse embedding} into Hilbert space if there exists
\begin{itemize}
\item a field of Hilbert spaces $\lbrace H_{x} \rbrace_{x \in X}$ over X;
\item a section $s: X \rightarrow \sqcup_{x \in X}H_{x}$ (i.e $s(x) \in H_{x}$);
\item two non-decreasing functions $\rho_{1}, \rho_{2}$ from $[0,\infty)$ to $(-\infty, +\infty)$ such that $\lim_{r\rightarrow \infty}\rho_{i}(r)=\infty$ for $i=1,2$.
\end{itemize}
such that for any $r>0$ there exists a bounded subset $K\subset X$ and a trivialisation:
\begin{equation*}
t_{C}: \lbrace H_{x} \rbrace_{x \in C} \rightarrow C\times H
\end{equation*}
for each $C \subset X \setminus K$ of diameter less than $r$. We ask that this map $t_{C}|_{x}=t_{C}(x)$ is an affine isometry from  $H_{x}$ to $H$, satisfying:
\begin{enumerate}
\item for any $x,y \in C$, $\rho_{1}(d(x,y))\leq \Vert t_{C}(x)(s(x)) - t_{C}(y)(s(y)) \Vert \leq \rho_{2}(d(x,y))$;
\item for any two subsets $C_{1},C_{2} \subset X\setminus K$ of diameter less than $r$ and nonempty intersection $C_{1}\cap C_{2}$ there exists an affine isometry $t_{C_{1},C_{2}}:H \rightarrow H$ such that $t_{C_{1}}(x)t_{C_{2}}(x)^{-1}=t_{C_{1},C_{2}}$ for all $x \in C_{1}\cap C_{2}$
\end{enumerate}
\end{definition}

In particular many expanders are known to satisfy this property, such as those coming from discrete groups with the Haagerup property  or those of large girth \cite{FCEpaper,MR2568691}.

The main application of this property is Theorem 1.1 \cite{FCEpaper}:

\begin{theorem}\label{Thm:FCEMR}
Let $X$ be a proper metric space admitting a fibred coarse embedding into Hilbert space. Then the maximal coarse Baum-Connes conjecture holds for $X$.\qed
\end{theorem}

We give a different proof of this in Section \ref{Sect:apps}.

\subsection{Groupoids and coarse properties.}\label{sect:coarsegroupoids}
Groupoids play an integral role in the constructions we adapt from \cite{MR1905840}. Below we recap the definition and basic properties of groupoids before giving an outline of the construction of the coarse groupoid associated to a coarse space $X$.
\begin{definition}\label{def:grpoid2}
A \textit{groupoid} is a set $\G$ equipped with the following information:
\begin{enumerate}
\item A subset $\G^{(0)}$ consisting of the objects of $\G$, denote the inclusion map by $i: \G^{(0)}\hookrightarrow \G$. 
\item Two maps, $r$ and $s: \G  \rightarrow \G^{(0)}$ such that $r\circ i = s \circ i = Id$ 
\item An involution map $^{-1}:\G \rightarrow \G$ such that $s(g)=r(g^{-1})$
\item A partial product $\G^{(2)} \rightarrow \G$ denoted $(g,h) \mapsto gh$, with $\G^{(2)}=\lbrace (g,h) \in \G \times \G | s(g)=r(h) \rbrace\subseteq \G\times \G$ being the set of composable pairs.
\end{enumerate}
Moreover we ask the following:
\begin{itemize}
\item The product is associative where it is defined in the sense that for any pairs: 
\begin{equation*}
(g,h),(h,k)\in \G^{(2)} \mbox{ we have }(gh)k \mbox{ and } g(hk) \mbox{ defined and equal}.
\end{equation*}
\item For all $g \in \G$ we have $r(g)g=gs(g)=g$.
\end{itemize}
\end{definition}

A groupoid $\G$ is a \textit{topological groupoid} if both $\G$ and $\G^{(0)}$ are topological spaces, and the maps $r,s, ^{-1}$ and the composition are all continuous. A Hausdorff, locally compact topological groupoid $\G$ is \textit{proper} if $(r,s)$ is a proper map and \textit{\'etale} or \textit{r-discrete} if the map $r$ is a local homeomorphism. When $\G$ is \'etale, $s$ and the product are also local homeomorphisms, and $\G^{(0)}$ is an open subset of $\G$.

Let $X$ be a uniformly discrete bounded geometry (sometimes denoted uniformly locally finite) metric space. We want to define a groupoid with property that it captures the coarse information associated to $X$. To do this effectively we need to define the what we mean by a \textit{coarse structure} that is associated to a metric. The details of this can be found in \cite{MR2007488}.

\begin{definition}
Let $X$ be a set and let $\mathcal{E}$ be a collection of subsets of $X \times X$. If $\mathcal{E}$ has the following properties:
\begin{enumerate}
\item $\mathcal{E}$ is closed under finite unions;
\item $\mathcal{E}$ is closed under taking subsets;
\item $\mathcal{E}$ is closed under the induced product and inverse that comes from the groupoid product on $X \times X$.
\item $\mathcal{E}$ contains the diagonal
\end{enumerate}
Then we say $\mathcal{E}$ is a \textit{coarse structure} on $X$ and we call the elements of $\mathcal{E}$ \textit{entourages}. If in addition $\mathcal{E}$ contains all finite subsets then we say that $\mathcal{E}$ is \textit{weakly connected}.
\end{definition}

\begin{example}\label{ex:MCS}
Let $X$ be a metric space. Then consider the collection $\mathcal{S}$ given by the $R$-neighbourhoods of the diagonal in $X\times X$; that is, for every $R>0$ the set:
\begin{equation*}
\Delta_{R}=\lbrace (x,y) \in X \times X | d(x,y)\leq R \rbrace
\end{equation*}
Then let $\mathcal{E}$ be the coarse structure generated by $\mathcal{S}$. This is called the \textit{metric coarse structure} on $X$. If  $X$ is a uniformly discrete bounded geometry  metric space this coarse structure is uniformly locally finite, proper and weakly connected.
\end{example}

To build a groupoid from the metric coarse structure on $X$ we consider extensions of the pair product on $X \times X$. The most natural way to do this is by making use of the entourages arising from the metric by enlarging them inside the universal compactification $\beta(X \times X)$. In order to get a suitable product on this object we utilise the following Lemma, which is Corollary 10.18 from \cite{MR2007488}:

\begin{lemma}\label{Lem:CorRoe} 
Let $X$ be a uniformly discrete bounded geometry metric space and let $E$ be any entourage. Then the inclusion $E \rightarrow X \times X$ extends to an injective homeomorphism $\overline{E} \rightarrow \beta X \times \beta X$, where $\overline{E}$ denotes the closure of $E$ in $\beta(X \times X)$.\qed
\end{lemma}
Now we can make the definition of the coarse groupoid $G(X)$:
\begin{theorem}(\cite[Theorem 10.20]{MR2007488})
Let $X$ be a coarse space with uniformly locally finite, weakly connected coarse structure $\mathcal{E}$. Define $G(X):=\cup_{E\in \mathcal{E}}\overline{E}.$ Then $G(X)$ is a locally compact, $\sigma$-compact, \'etale groupoid with the induced product, inverse and topology from $\beta X \times \beta X$.\qed
\end{theorem}
As we are considering the metric coarse structure we can reduce this to considering only generators:
\begin{equation*}
G(X):=\bigcup_{R>0}\overline{\Delta_{R}}
\end{equation*}

The following Theorem connects the coarse geometry of $X$ to properties of $G(X)$.

\begin{theorem}\label{thm:coarseprop}
Let $X$ be a uniformly discrete space with bounded geometry and let $G(X)$ be its coarse groupoid. Then the following hold:
\begin{enumerate}
\item $X$ has property A if and only if $G(X)$ is amenable \cite[Theorem 5.3]{MR1905840};
\item $X$ admits a coarse embedding into Hilbert space if and only if $G(X)$ has the Haagerup property \cite[Theorem 5.4]{MR1905840}.\qed
\end{enumerate}
\end{theorem}

The latter aspect of Theorem \ref{thm:coarseprop} is characterised by the groupoid $G(X)$ admitting a proper negative type function to $\mathbb{R}$, which Tu proved was equivalent to a proper affine action on Hilbert space \cite{MR1703305}.

\section{Negative type functions on groupoids.}\label{sect:negtype}
The role of positive and conditionally negative type kernels within group theory is well known and plays an important role in studying both anayltic and representation theoretic properties of groups \cite{MR2415834,MR1487204}. These ideas were extended to groupoids by Tu \cite{MR1703305}. Let $\G$ be a locally compact, Hausdorff groupoid.

\begin{definition}
A continuous function $F: \G \rightarrow \mathbb{R}$ is said to be of \textit{negative type} if 
\begin{enumerate}
\item $F|_{\G^{(0)}}=0$;
\item $\forall x \in \G, F(x)=F(x^{-1})$;
\item Given $x_{1},...,x_{n} \in \G$ all having the same range and $\sigma_{1},...,\sigma_{n} \in \mathbb{R}$ such that $\sum_{i}\sigma_{i}=0$ we have $\sum_{j,k}\sigma_{j}\sigma_{k}F(x_{j}^{-1}x_{k})\leq 0$.
\end{enumerate}
\end{definition}

The important feature of functions of this type is their connection to the Haagerup property for locally compact, $\sigma$-compact groupoids, in fact the following are equivalent \cite{MR1703305}:
\begin{enumerate}
\item There exists a proper negative type function on $\G$
\item There exists a continuous field of Hilbert spaces over $\G^{(0)}$ with a proper affine action of $\G$.
\end{enumerate}

\section{Main Theorem}
We dedicate this section to proving Theorem \ref{Thm:MR1}.

\subsection{The new groupoid.}

In this section we outline the construction of the groupoid $\mathcal{M}_{\mathcal{F}}(X)$ associated with a family of bounded subsets $\mathcal{F}$ of $X$. We then outline the construction of a negative type function on this groupoid provided a fibred coarse embedding into Hilbert space. This will prove one direction of Theorem \ref{Thm:MR1}.

\begin{definition}
Let $\mathcal{F}:=\lbrace K_{R} \rbrace$ be a family of bounded subsets of X. Let $A_{R}$ be restricted entourage $\Delta_{R}\cap ((X\setminus K_{R}) \times (X\setminus K_{R}))$. Then we denote by $\mathcal{M}_{\mathcal{F}}(X)$ the set $\bigcup_{R>0} \overline{A}_{R}$, where $\overline{A}_{R}$ is the closure in $\beta (X\times X)$.
\end{definition}

\begin{lemma}
$\mathcal{M}_{\mathcal{F}}(X)$ is a groupoid with operations induced from the pair groupoid operation defined where it makes sense; i.e:
\begin{equation*}
(x,y)(y,z)=(x,z)
\end{equation*}
if there exists $S_{1},S_{2},S_{3}>0$ such that  $(x,y) \in A_{S_{1}},(y,z) \in A_{S_{2}}$ and $(x,z) \in A_{S_{3}}$.  
\end{lemma}
\begin{proof}
The fact that composition is well defined, as well as the other groupoid axioms is technical but clear. The fact it passes to completions relies on the fact that nets with cofinite overlap have the same limit.
\end{proof}

\begin{rem}\label{rem:outline}
Let $X$ admit a fibred coarse embedding into Hilbert space.
\begin{enumerate}
\item $A_{R}:=\Delta_{R}(X\setminus K_{R})$ is an entourage with the same corona as $\Delta_{R}$. Assume that $X$ admits a fibred coarse embedding into Hilbert space with respect to $\mathcal{F}$. Then for each $R>0$ such that and the following function is defined on $A_{R}$:
\begin{equation*}
k_{R}(x,y)=\Vert t_{x}(x)(s(x))-t_{x}(y)(s(y))\Vert^{2}
\end{equation*}
where $t_{x}:H_{x} \rightarrow \mathcal{H}$ is the affine isometry provided by the fibred coarse embedding.
\item In the special case that $X$ is a space of graphs $\sqcup X_{i}$, the $A_{R}$'s defined above have the form: $A_{R}= \sqcup_{i\geq i_{R}}\Delta_{R}^{i}$ for some $i\geq i_{R}$.
\item It is clear that $G(X)|_{\partial\beta X}$ is the boundary groupoid of $\mathcal{M}_{\mathcal{F}}(X)$.
\item As $G(X)|_{\partial\beta X} = \bigcup_{R>0} \overline{\Delta_{R}}\setminus \Delta_{R}$ topologically, we know that for any $\gamma \in G(X)|_{\partial\beta X}$ that there is a smallest $R>0$ such that $\gamma \in \overline{\Delta_{R}}\setminus \Delta_{R}$. 

\end{enumerate}
\end{rem}

We now extend each $k_{R}$ to the boundary and show that they fit together to define a global function $f$ that is well-defined on $\mathcal{M}_{\mathcal{F}}(X)$.

\begin{proposition}\label{prop:wd}
There is a globally defined function $f$ that is constructed from the $\tilde{k}_{R}$. The function $f$ is well defined and continuous on $\mathcal{M}_{\mathcal{F}}(X)$.
\end{proposition}
\begin{proof}
We have two considerations:
\begin{enumerate}
\item Extending $k_{R}$ to $\tilde{k}_{R}$ on the closure of $A_{R}$ for each $R>0$.
\item Each $\tilde{k}_{R}$ pieces together; for any $S>R$ we have that $\tilde{k}_{S}(\gamma)=\tilde{k}_{R}(\gamma)$.
\end{enumerate}
We first prove (1). Under the assumption that $X$ fibred coarsely embeds into Hilbert space, we know that each $k_{R}$ is a bounded function on $A_{R}$, hence extends to a continuous function $\tilde{k}_{R}$ on the Stone-Cech compactification of $A_{R}$, which in this context is homeomorphic to its closure in $\beta(X \times X)$. 

For the proof of (2) consider the sets $A_{R}$ and $A_{R}^{S}:=A_{R}\cap A_{S}$. Using the compatibility properties of a fibred coarse embedding we can see that the function $k_{R}$, and $k_{S}$ restricted to $A_{R}^{S}$ agree; consider that for any $(x,y)\in A_{R}^{S}$ there is a unique affine isometry $t_{B}$ such that $t_{B}t_{B_{S}}(x)(x) = t_{B_{R}}(x)$. As $k_{R}$ is preserved by affine isometries for all $R>0$ and as $A_{R}^{S}$ and $A_{R}$ have the same corona, we have that $\tilde{k}_{R}$ and $\tilde{k}_{S}$ restricted to $\overline{A_{R}}\setminus A_{R}$ agree. Hence we can define, for any $\gamma \in \mathcal{M}_{\mathcal{F}}(X)$, $f(\gamma) = \tilde{k}_{R_{\gamma}}(\gamma)$, which is the natural continuous function defined on the union $\cup_{R>0}\overline{A_{R}}$.
\end{proof}


\begin{lemma}\label{lem:MT1-a}
The function $f$ is proper.
\end{lemma}
\begin{proof}
To see $f$ is proper it is enough to prove that the preimage of an interval $[0,r]$ is contained in $\overline{A_{R}}$ for some $R>0$. This is as each interval $[0,r]$ is a closed subset of $\mathbb{R}$, the map $f$ is continuous and hence $f^{-1}([0,r])$ would be a closed subset of a compact set, hence would itself be compact.

We now assume for a contradiction that the preimage of $[0,r]$ contains elements $\gamma$ with the $R$ given by Remark \ref{rem:outline}.(1) being arbitrarily large. In this proof we denote that $R$ by $R_{\gamma}$. Then from the definition of $f$ and the fact that $X$ admits a fibred coarse embedding, we can see:
\begin{equation*}
\rho_{-}(R_{\lambda})^{2} \leq f(\gamma_{\lambda}) \leq \rho_{+}(R_{\lambda})^{2}.
\end{equation*}
As $\rho_{-}(S)$ tends to infinity if $S$ does, we can find an $S>0$ such that $\rho_{-}(S)>r$. By assumption there exists $\gamma \in f^{-1}([0,r])$ with $R_{\gamma}$ as large as we like, in particular $R_{\gamma}>S$, which is impossible. Whence, there exists an $R>0$ such that $f^{-1}([0,r]) \subset \overline{A_{R}}$.
\end{proof}


\begin{lemma}\label{lem:MT1-b}
The function $f$ is of negative type.
\end{lemma}
\begin{proof}
This relies on the ideas of \cite[Theorem 5.4]{MR1905840}, and it is enough to check that this holds for the boundary subgroupoid $G(X)|_{\partial\beta X}$ of $\mathcal{M}_{\mathcal{F}}(X)$, as it clearly holds for its compliment. Let $\gamma_{1},...,\gamma_{n} \in G(X)|_{\partial\beta X}$ such that $r(\gamma_{1})=...=r(\gamma_{n}):=\omega$ and let $\sigma_{1},...,\sigma_{n} \in \mathbb{R}$ with sum $0$. We need to prove that:
\begin{equation*}
\sum_{i,j}\sigma_{i}\sigma_{j}f(\gamma_{i}^{-1}\gamma_{j}) \leq 0.
\end{equation*}
As there are only finitely many $\gamma_{i}$, there exists a smallest $R>0$ such that each $\gamma_{i}$ and each product $\gamma^{-1}_{j}\gamma_{i}$ are elements of $\overline{\Delta_{R}}$. Let $(x_{i,\lambda},y_{i,\lambda})$ be nets within $\Delta_{R}$ that converge to $\gamma_{i}$ respectively. As the ranges of the $\gamma_{i}$ are all equal, we know that $y_{\lambda_{i}}\rightarrow \omega$ for each $i$, so we can assume without loss of generality that $y_{\lambda,i}=y_{\lambda}$ is equal in each net. Hence, for each $\lambda$, we know that $(x_{\lambda,i},y_{\lambda})$ and $(x_{\lambda,j},x_{\lambda,i}) \in \Delta_{S}$, and that $x_{\lambda,i}, x_{\lambda,j}\in B_{S}(y_{\lambda})$.

Now we compute, relative to $y_{\lambda}$.
\begin{eqnarray*}
&&\sum_{i,j}\sigma_{i}\sigma_{j}k_{S}^{y_{\lambda}}(x_{\lambda, j},x_{\lambda, i}) = \sum_{i,j}\sigma_{i}\sigma_{j}\Vert t_{y_{\lambda}}(x_{\lambda ,j})(s(x_{\lambda ,j})) - t_{y_{\lambda}}(x_{\lambda ,i})(s(x_{\lambda ,i})) \Vert^{2}\\
&&= \sum_{i,j}\sigma_{i}\sigma_{j}(\Vert t_{y_{\lambda}}(x_{\lambda ,j})(s(x_{\lambda ,j})) \Vert^{2} + \Vert t_{y_{\lambda}}(x_{\lambda ,i})(s(x_{\lambda ,i})) \Vert^{2} - 2 \langle t_{y_{\lambda}}(x_{\lambda ,j})(s(x_{\lambda ,j})), t_{y_{\lambda}}(x_{\lambda ,i})(s(x_{\lambda ,i}))\rangle)\\
&&=(\sum_{j}\sigma_{j}\Vert t_{y_{\lambda}}(x_{\lambda ,j})(s(x_{\lambda ,j})) \Vert^{2})(\sum_{i}\sigma_{i})+(\sum_{i}\sigma_{i}\Vert t_{y_{\lambda}}(x_{\lambda ,i})(s(x_{\lambda ,i})) \Vert^{2})(\sum_{j}\sigma_{j})\\
&&  -2\langle \sum_{j}\sigma_{j}t_{y_{\lambda}}(x_{\lambda ,j})(s(x_{\lambda ,j})),\sum_{i}\sigma_{i}t_{y_{\lambda}}(x_{\lambda ,i})(s(x_{\lambda ,i}))\rangle \leq 0.
\end{eqnarray*}
This holds for each $\lambda$ in the net. Taking a limit in $\lambda$:
\begin{equation*}
\sum_{i,j}\sigma_{i}\sigma_{j}f(\gamma_{i}^{-1}\gamma_{j})=\sum_{i,j}\sigma_{j}\sigma_{i}\lim_{\lambda}k_{S}^{y_{\lambda}}(x_{\lambda ,j},x_{\lambda, i}) \leq 0
\end{equation*}
\end{proof}

The preceding Lemmas prove the following Theorem, which is one half of Theorem \ref{Thm:MR1}.

\begin{theorem}\label{Thm:MT2}
Let $X$ be a uniformly discrete space with bounded geometry that admits a fibred coarse embedding into Hilbert space with respect to the family $\mathcal{F}:=\lbrace K_{R} \rbrace_{R>0}$. Then there is a proper conditionally negative type function defined on $\mathcal{M}_{\mathcal{F}}(X)$.\qed
\end{theorem}

We immediately get the following corollary, which is important for the applications of Section \ref{sect:apps}.

\begin{corollary}\label{thm:MT1-a}
Let $X$ as above. Then the boundary groupoid $G(X)|_{\partial\beta X}$ admits a proper negative type function to $\mathbb{R}$.\qed
\end{corollary}

\subsection{The other half of Theorem \ref{Thm:MR1}.}

In this section we prove the converse to Theorem \ref{Thm:MT2}. Fix $\mathcal{F}:=\lbrace K_{R} \rbrace$ as a family of finite subsets of $X$.

\begin{proposition}
If $\mathcal{M}_{\mathcal{F}}(X)$ has the Haagerup property, then $X$ admits a fibred coarse embedding into Hilbert space with respect to $\mathcal{F}$.
\end{proposition} 
\begin{proof}
Under the assumption of the Haagerup property, as outlined in Section \ref{sect:negtype}, it is known that the groupoid $\mathcal{M}_{\mathcal{F}}(X)$ admits a continous, proper, isometric affine action on a Hilbert bundle over $\mathcal{M}_{\mathcal{F}}(X)^{(0)}$ \cite{MR1703305}. Without loss, we can also assume that $X \subset \mathcal{M}_{\mathcal{F}}(X)^{(0)}$. To construct a fibred coarse embedding we first construct the section:
\begin{equation*}
s(x) = 0_{x} \in H_{x}
\end{equation*}
Now we construct the trivialisation for each $R>0$. Consider $C$, a subset of $X\setminus K_{R}$ of diameter at most $R$ by picking a basepoint $x_{C} \in C$ then using the isometric affine action of $\mathcal{M}_{\mathcal{F}}(X)$:
\begin{equation*}
t_{C}(x)=\alpha(x_{C},x)
\end{equation*}
Now it follows that $\Vert t_{C}(x)(s(x)) - t_{C}(y)(s(y)) \Vert = \Vert \alpha(y,x)0_{x} \Vert$ as the groupoid acts by affine isometries. Given that $\gamma \rightarrow \Vert \alpha(\gamma)0_{r(\gamma)} \Vert$ is a continous proper map and the closures of the $A_{R}$ are compact we can deduce that property (1) from Definition \ref{def:FCE} holds for any $C$. 

From here, we are left to check property (2) of Definition \ref{def:FCE}. Let $C_{1}$ and $C_{2}$ be sets of diameter at most $R>0$ and nonempty intersection, then choose a basepoints $x_{0}, x_{1}$ for $C_{1}$ and $C_{2}$ respectively. We can assume without loss of generality that $x_{i}$ belong to the intersection, whence the transform $\alpha(x_{1},x_{0})$ arising from the affine action of $\mathcal{M}_{\mathcal{F}}(X)$ proves the affine isometry that changes $t_{C_{1}}$ into $t_{C_{2}}$ everywhere on the intersection. 
\end{proof}

We have now completed the proof of the following analogue of Theorem 5.4 of \cite{MR1905840}:

\begin{theorem}
Let $X$ be a uniformly discrete metric space of bounded geometry and let $\mathcal{F}:=\lbrace K_{R} \rbrace$ be a family of bounded subsets of $X$. Then the following are equivalent:
\begin{enumerate}
\item $X$ admits a fibred coarse embedding into Hilbert space with respect to $\mathcal{F}$;
\item The groupoid $\mathcal{M}_{\mathcal{F}}(X)$ has the Haagerup property. \qed
\end{enumerate}
\end{theorem}

\section{Applications}\label{sect:apps}

\subsection{The boundary coarse Baum-Connes conjecture and the coarse Novikov conjecture.}\label{Sect:apps}
Recall from \cite{mypub1} the boundary coarse Baum-Connes conjecture.
\begin{conjecture1} [Boundary Coarse Baum-Connes Conjecture]
Let $X$ be a uniformly discrete bounded geometry metric space. Then the assembly map:
\begin{equation*}
\mu_{bdry}:K_{*}^{top}(G(X)|_{\partial\beta X}, l^{\infty}(X,\mathcal{K})/C_{0}(X,\mathcal{K})) \rightarrow K_{*}((l^{\infty}(X,\mathcal{K})/C_{0}(X,\mathcal{K}))\rtimes_{r}G(X)|_{\partial\beta X})
\end{equation*}
is an isomorphism.
\end{conjecture1}

This conjecture also has a maximal form \cite[Section 2]{mypub1} and that is equivalent to the maximal coarse Baum-Connes conjecture at infinity from \cite{FCEpaper},if $X$ is a uniformly discrete bounded geometry metric space we can see that the algebra at infinity defined in \cite{FCEpaper} and the groupoid crossed product algebra $\ell^{\infty}(X,\mathcal{K})\rtimes_{m}G(X)|_{\partial\beta X}$ are isomorphic. Proceeding via this conjecture we can appeal the machinery of Tu \cite{MR1703305} concerning $\sigma$-compact, locally compact groupoids with the Haagerup property to conclude results about the (maximal) coarse Baum-Connes conjecture. 

In particular, we can use this conjecture and homological algebra to conclude Theorem \ref{Thm:FCEMR}.

\begin{theorem}
Let $X$ be a uniformly discrete space with bounded geometry that fibred coarse embeds into Hilbert space. Then the maximal coarse Baum-Connes assembly map is an isomorphism for $X$.
\end{theorem}
\begin{proof}
We have a short exact sequence of groupoid $C^{*}$-algebras:
\begin{equation*}
0 \rightarrow \mathcal{K} \rightarrow C^{*}_{max}(G(X)) \rightarrow C^{*}_{max}(G(X)|_{\partial\beta X}) \rightarrow 0.
\end{equation*}
This gives us the following diagram, arising from the long exact sequence in K-theory and suitable Baum-Connes conjectures (omitting the coefficients):
\begin{equation*}
\xymatrix@=0.7em{
K_{1}(C^{*}(G(X)|_{\partial\beta X}) \ar[r] & K_{0}(\mathcal{K}) \ar[r]& K_{0}(C^{*}(G(X))) \ar[r]& K_{0}(C^{*}(G(X)|_{\partial\beta X})\ar[r] & K_{1}(\mathcal{K})  \\
K_{1}^{top}(G(X)|_{\partial\beta X}) \ar[r] \ar[u]& K_{0}^{top}(X \times X) \ar[r]\ar[u]^{\ucong}& K_{0}^{top}(G(X)) \ar[r]\ar[u]& K_{0}^{top}(G(X)|_{\partial\beta X}) \ar[r]\ar[u]^{\mu_{bdry}}& K_{1}^{top}(X \times X)\ar[u]^{\ucong}
}
\end{equation*}
By Corollary \ref{Thm:MT1} the maximal boundary assembly map is an isomorphism. The result now follows from the Five lemma.
\end{proof}

To understand the reduced assembly map requires a more delicate approach.

\begin{definition}
Let $X$ as above. We say that $X$ has an \textit{infinite coarse component} if there exists $E \in \mathcal{E}$ such that $P_{E}(X)$ has an infinite connected component. Otherwise we say that $X$ \textit{only has finite coarse components}. In the metric coarse structure, this condition becomes: if there exists an $R>0$ such that $P_{R}(X)$ has an infinite connected component.
\end{definition}

\begin{lemma}\label{lem:zandi}
Let $X$ be a uniformly discrete bounded geometry metric space, $i: \mathcal{K} \hookrightarrow C^{*}X$ be the inclusion of the compact operators into the Roe algebra of $X$ and $j:X\times X \rightarrow G(X)$. Then:
\begin{enumerate}
\item If $X$ has an infinite coarse component then the map from $K^{top}(X \times X) \rightarrow K^{top}(G(X))$ is the $0$ map.
\item If $X$ has only finite coarse components then $i_{*}:K_{*}(\mathcal{K}) \rightarrow K_{*}(C^{*}X)$ is injective.
\end{enumerate}
\end{lemma}
\begin{proof}
Without loss of generality we state the proof for graphs. 

Claim (1) relies on an argument involving groupoid equivariant KK-theory. We consider the following diagram, with notation following \cite{MR1905840,MR1656031}, where $P_{E}(G(X))$ is the closure of $P_{E}(X)\times X$ in the weak $*$-topology on the dual of $C_{c}(G(X))$:
\begin{equation*}
\xymatrix{KK_{X\times X}(C_{0}(U),C_{0}(X,\mathcal{K})) \ar[r]_{1} & KK_{G(X)}(C_{0}(P_{E}(G(X))),\ell^{\infty}(X,\mathcal{K})) \ar[r]^{\cong}_{4}  & KK(C_{0}(P_{E}(X)),\mathbb{C}) \\
& KK(\mathbb{C},\mathbb{C}) \ar^{\cong}_{2}[lu]\ar[u]_{3}\ar[ur]_{5}&
}
\end{equation*}
If this diagram commutes, then we can conclude the result in two steps. First, we take a limit through the directed set of entourages $E \in \mathcal{E}$ to get:
\begin{equation*}
\xymatrix{K^{top}(X \times X, C_{0}(X,\mathcal{K})) \ar[r] & K^{top}(G(X)),\ell^{\infty}(X,\mathcal{K})) \ar[r]^{\cong}  & KX_{*}(X) \\
& KK(\mathbb{C},\mathbb{C}) \ar^{\cong}[lu]\ar[u]\ar[ur]&
}
\end{equation*}
Secondly, having an infinite coarse component in $X$ allows us to conclude that the map from $KK(\mathbb{C},\mathbb{C})$ factors through the coarse K-homology of a ray, and hence is $0$.

We now prove the first diagram commutes. Consider the maps:
\begin{eqnarray*}
1 & : & (E,\phi , F) \mapsto (E\otimes_{\rho}\ell^{\infty}(X,\mathcal{K}),\phi \otimes 1 , F \otimes 1)\\
2 & : & (\mathbb{C},1,0) \mapsto (C_{0}(X),1,0)\\
3 & : & (\mathbb{C},1,0) \mapsto (C_{0}(P_{E}(G(X))),1,0)\\
4 & : & (E,\phi , F) \mapsto (E|_{x},\phi|_{x},F|_{x})\\
5 & : & (\mathbb{C},1,0) \mapsto (C_{0}(P_{E}(X),1,0)
\end{eqnarray*}
As $KK_{G(X)}(C_{0}(Z),B) \cong KK_{X\times X}(C_{0}(Z),B)$ for any proper $X\times X$-space $Z$ and any $X \times X$-algebra $B$ and $U$ is $X\times X$ homotopy equivalent to $X$, we can conclude that the triangle formed from maps $1$,$2$ and $3$ commutes. The map $4$ is induced by the inclusion of $\lbrace x \rbrace$ into $G(X)$ as a subgroupoid. Applying $4$ to the image of the generator $(\mathbb{C},1,0)$ of $KK(\mathbb{C},\mathbb{C})$ we obtain the image of the generator under $5$. Hence the second triangle commutes. This completes the proof of Claim (1).

Claim (2) in the context of graphs reduces to studying sequences of finite graphs equipped with a coarsely disjoint box metric. In this context, this is Proposition 2.9 of \cite{mypub1}. We give a proof below.

Consider the subalgebra $C^{*}X_{\infty}$ - the Roe algebra of the space of graphs with the disjoint union 'metric'. In this case, we get a diagram:
\begin{equation*}
\xymatrix{\bigoplus_{i \in \mathbb{N}} M_{i} \ar@{->}[r]\ar@{=}[d]  & C^{*}X_{\infty} \ar@{->}[r]& \prod_{i\in \mathbb{N}}M_{i}\ar@{=}[d]\\
\bigoplus_{i \in \mathbb{N}} M_{i}\ar@{^{(}->}[rr] & & \prod_{i\in \mathbb{N}}M_{i}
}
\end{equation*}
with the long inclusion inducing an inclusion on K-theory. Hence, the map $\bigoplus_{i \in \mathbb{N}} M_{i} \rightarrow C^{*}X_{\infty}$ is injective. We now compare this with the inclusion $\mathcal{K} \rightarrow C^{*}X$ on K-theory using the following diagram:
\begin{equation*}
\xymatrix{K_{1}(\frac{C^{*}X_{\infty}}{\bigoplus_{i\in\mathbb{N}}M_{i}})\ar@{->}[r]^{0}\ar@{=}[d]& K_{0}(\bigoplus_{i\in \mathbb{N}}M_{i}) \ar@{^{(}->}[r]\ar@{->>}[d] & K_{0}(C^{*}X_{\infty}) \ar@{->>}[r] \ar@{->}[d] & K_{0}(\frac{C^{*}X_{\infty}}{\bigoplus_{i\in\mathbb{N}}M_{i}})\ar@{=}[d]\\
K_{1}(\frac{C^{*}X}{\mathcal{K}})\ar@{->}[r]& K_{0}(\mathcal{K}) \ar@{->}[r] & K_{0}(C^{*}X) \ar@{->>}[r]  & K_{0}(\frac{C^{*}X}{\mathcal{K}})}
\end{equation*}
A diagram chase will now prove the desired result.
\end{proof}

\begin{theorem}\label{thm:mcor1}
Let $X$ be a uniformly discrete bounded geometry metric space that admits a fibred coarse embedding into Hilbert space. Then the coarse Baum-Connes assembly map for $X$ is injective.
\end{theorem}
\begin{proof}
For a uniformly discrete space $X$ with bounded geometry we know that the ghost ideal $I_{G}$ fits into the sequence:
\begin{equation*}
0 \rightarrow I_{G} \rightarrow C^{*}X \rightarrow (\frac{\ell^{\infty}(X,\mathcal{K})}{C_{0}(X,\mathcal{K})})\rtimes G(X)|_{\partial\beta X} \rightarrow 0.
\end{equation*}
This gives rise to the ladder in K-theory and K-homology where the rungs are the assembly maps defined in \cite{mypub1}:
$$
\xymatrix@=1em{
\ar[r] & K_{1}(C^{*}_{r}(G(X)|_{\partial\beta X}) \ar[r] & K_{0}(I_{G}) \ar[r]& K_{0}(C^{*}_{r}(G(X))) \ar[r]& K_{0}(C^{*}_{r}(G(X)|_{\partial\beta X})\ar[r] & K_{1}(I_{G}) \ar[r] & \\
\ar[r] & K_{1}^{top}(G(X)|_{\partial\beta X}) \ar[r] \ar[u]^{\mu^{bdry}}& K^{top}(X \times X) \ar[r]_{1}\ar[u]_{2}& K_{0}^{top}(G(X)) \ar[r]\ar[u]^{\mu}& K_{0}^{top}(G(X)|_{\partial\beta X}) \ar[r]\ar[u]^{\mu^{bdry}}& 0\ar[u]^{0} \ar[r] &
}
$$
Corollary \ref{Thm:MT1} allows us to conclude that $\mu^{bdry}$ is an isomorphism. Now we treat cases. If $X$ has an infinite coarse component then Lemma \ref{lem:zandi}.(1) implies the map $K^{top}(X\times X) \rightarrow K^{top}(G(X))$ is the zero map. Now assume that $x \in K^{top}(G(X))$ maps to $0$ in $K_{0}(C^{*}_{r}(G(X))$. Then it maps to $0$ in $K^{top}(G(X)|_{\partial\beta X})$ as $\mu^{bdry}$ is an isomorphism. As the second line is exact, this implies it comes an element in $K^{top}(X\times X)$. As the map labelled $1$ is the zero map, $x$ must be $0$.

If $X$ has only finite coarse components, then by Lemma \ref{lem:zandi} the map $2$ is injective and so we can conclude injectivity of $\mu$ by the Five Lemma.
\end{proof}

We can also describe the obstructions to $\mu_{*}$ being an isomorphism when $X$ admits a fibred coarse embedding into Hilbert space.

\begin{proposition}\label{thm:MT2}
Let $X$ be a uniformly discrete metric space with bounded geometry such that $X$ admits a fibred coarse embedding into Hilbert space. Then the inclusion of $\mathcal{K}$ into $I_{G}$ induces an isomorphism on K-theory if and only if $\mu_{*}$ is an isomorphism. In addition, if $X$ has only finite coarse components then every ghost projection in $C^{*}X$ is compact if and only if $\mu_{0}$ is an isomorphism.
\end{proposition}
\begin{proof}
We consider the diagram from the proof of Theorem \ref{thm:mcor1}
$$
\xymatrix@=1em{
\ar[r] & K_{1}(C^{*}_{r}(G(X)|_{\partial\beta X}) \ar[r] & K_{0}(I_{G}) \ar[r]& K_{0}(C^{*}_{r}(G(X))) \ar[r]& K_{0}(C^{*}_{r}(G(X)|_{\partial\beta X})\ar[r] & K_{1}(I_{G}) \ar[r] & \\
\ar[r] & K_{1}^{top}(G(X)|_{\partial\beta X}) \ar[r] \ar[u]^{\mu^{bdry}}& K^{top}(X \times X) \ar[r]_{1}\ar[u]_{2}& K_{0}^{top}(G(X)) \ar[r]\ar[u]^{\mu}& K_{0}^{top}(G(X)|_{\partial\beta X}) \ar[r]\ar[u]^{\mu^{bdry}}& 0\ar[u]^{0} \ar[r] &
}
$$
When $X$ fibred coarsely embeds into Hilbert space, we know that $\mu^{bdry}$ is an isomorphism. The result then follows from the Five Lemma.

Now assume $X$ has only finite coarse components: it is enough to consider just sequences of finite graphs $\lbrace X_{i}\rbrace$ and in this case we have access to a tracelike map defined in \cite{higsonpreprint, explg1} given as follows.

For any $T \in C^{*}X$ we have a natural decomposition $T= T_{0} \sqcup \bigsqcup_{i>i_{R}}T_{i}$, where each $T_{i} \in C^{*}X_{i}$. We define a map $\frac{\prod_{i}C^{*}X_{i}}{\oplus C^{*}X_{i}}$ that sends each $T$ to $\prod_{i>i_{R}} T_{i}$. This induces a map on K-theory that we denote by $Tr$.
\begin{eqnarray*}
Tr:K_{0}(C^{*}X) \rightarrow \frac{\prod_{i}K_{0}(C^{*}X_{i})}{\oplus_{i}K_{0}(C^{*}X_{i})}.
\end{eqnarray*}
Under this trace a non-compact ghost projection will be non-zero but a compact will be $0$. If $\mu$ is an isomorphism, it follows that every ghost projection is equivalent to a compact on K-theory, that is $K_{*}(\mathcal{K}) \cong K_{*}(I_{G})$. Hence, any ghost projection in $C^{*}(X)$ vanishes under $Tr$, which happens if and only if the ghost projection is compact \cite{explg1}.
\end{proof}

A natural corollary of this concerns coarse embeddings, which partially answers a question initially raised by Roe \cite[Chapter 11]{MR2007488}:

\begin{corollary}\label{thm:IT3}
If $X$ coarsely embeds into Hilbert space then $K_{*}(I_{G}) \cong K_{*}(\mathcal{K})$. \qed
\end{corollary}

Additionally Proposition \ref{thm:MT2} implies that for the spaces of graphs that coarsely embed into Hilbert space but do not have property A \cite{MR2899681,MR2920843} it is not possible to find a ghost projection that is non-compact inside the Roe algebra. 

\subsection{Box spaces of residually finite discrete groups.}
The last application of the main result of this paper concerns box spaces of finitely generated residually finite groups. We show that the box space admitting a fibred coarse embedding into Hilbert space characterises the Haagerup property for the group in this instance. 

Let $\Gamma$ be a finitely generated residually finite discrete group, and let $\mathcal{N}:=\lbrace N_{i} \rbrace_{i\in\mathbb{N}}$ be a family of nested finite index subgroups with trivial intersection. Fix a generating set $S$ for $\Gamma$. Then we can construct a metric space $\square \Gamma$, called the \textit{box space} of $\Gamma$ with respect to the family $\mathcal{N}$ by considering the space of graphs of the sequence: $\lbrace Cay(\frac{\Gamma}{N_{i}},S) \rbrace_{i}$. 

It is well known \cite[Proposition 11.26]{MR2007488} that a coarse embedding of $\square \Gamma$ into Hilbert space implies that $\Gamma$ has the Haagerup property. Using Theorem 2.2 from \cite{FCEpaper}, it is possible to show that if $\Gamma$ has the Haagerup property, then any box space of $\Gamma$ has a fibred coarse embedding. The following Lemma will allow us to prove the converse of Theorem 2.2 from \cite{FCEpaper} using Corollary \ref{Thm:MT1}.

\begin{lemma}\label{lem:cor2}
Let $\square \Gamma$ be a box space of a finitely generated residually finite group $\Gamma$. Then the boundary $\partial\beta X$ admits a $\Gamma$-invariant Borel probability measure.
\end{lemma}
\begin{proof}
We work via linear functionals on $C(\beta X)$; consider the function:
\begin{equation*}
\mu(f)=\lim \frac{1}{\vert X_{i} \vert}\sum_{x \in X_{i}}f(x_{i}).
\end{equation*}
Clearly, $\mu(1_{\beta X})=1$ and $\mu$ is linear, whence $\mu$ is a linear functional. Let $g \in \Gamma$, now we check invariance:
\begin{equation*}
\mu(g\circ f)=\lim \frac{1}{\vert X_{i} \vert}\sum_{x \in X_{i}}f(gx_{i}).
\end{equation*}
After relabelling the elements $x \in X_{i}$ by $g^{-1}x^{'}$, we now see that $\mu(f)=\mu(g \circ f)$. The result now follows.
\end{proof}

It is well known that the boundary groupoid associated to a box space $\square\Gamma$ decomposes as $\partial\beta \square \Gamma \rtimes \Gamma$, for a proof see \cite{mypub1}. We now recall Corollary 5.12 of \cite{BG-action-2012}:

\begin{proposition}\label{prop:cor}
Let $\Gamma$ be a discrete group acting on a space $X$ with an invariant probablity measure $\mu$. Then the action is a-T-menable if and only if $\Gamma$ is a-T-menable.\qed
\end{proposition}

Now Lemma \ref{lem:cor2} plus the above Proposition prove:

\begin{theorem}\label{thm:cor2}
If the box space $\square \Gamma$ admits a fibred coarse embedding then the group $\Gamma$ has the Haagerup property.\qed
\end{theorem}


\bibliography{ref.bib}
\end{document}
